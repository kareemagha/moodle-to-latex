\documentclass[addpoints,12pt, margin-left=35px]{exam}
\usepackage{multicol}
\usepackage{tikz}
\usepackage{circuitikz}

\usepackage{graphicx} % Required for inserting images
\usepackage{amsmath}
\setlength{\parindent}{0pt}

\begin{document}

\textbf{Question:}\\

Two charges $+Q = +{9.51} \mu C$ and $-Q = - {9.51} \mu C$ are separated by a distance $a = {8.99} \text{ cm.}$ A third charge $ q =+ {2.80} \mu C$ is placed above these two charges to form an equilateral triangle as shown in the figure. Use vector notation to answer these questions.\\

\begin{center}
\includegraphics[width=0.8\textwidth]{images/1734050054302_0.png}
\end{center}\\

Figure 1. Three charges, $+q$, $+Q$ and $-Q$ are arranged in an equilateral triangle with side length $a$.\\

\textbf{Part 1)}\\

What force does $+ Q$ exert on $+q?$\\

$\mathbf{\vec{F}} = $  \underline{\hspace{3cm}}  $ \mathbf{\hat{i}} + $  \underline{\hspace{3cm}}  $\mathbf{\hat{j}} \text{ N}$ \\

\\

\textbf{Part 2)}\\

What is the net force on the $+ q$ charge?\\

$\mathbf{\vec{F}} = $  \underline{\hspace{3cm}}  $ \mathbf{\hat{i}} + $  \underline{\hspace{3cm}}  $\mathbf{\hat{j}} \text{ N}$ \\

\\

\textbf{Part 3)}\\

The $+q$ charge is now removed from the top of the triangle. What is the electric field at the top of the triangle now?\\

$\mathbf{\vec{E}} = $  \underline{\hspace{3cm}}  $ \mathbf{\hat{i}} + $  \underline{\hspace{3cm}}  $\mathbf{\hat{j}} \text{ N/C}$ \\

\\

\newpage

\textbf{Question:}\\

A non conducting rod forms an arc with radius $a $ subtending an angle $\theta $. It has a constant charge per unit length of $\lambda $ In this question you are going to derive an expression for the electric field at the centre of the circle the arc lies upon (the point in the figure). Use unit vector notation (the unit vectors shown in the figure). Use Coulomb's constant in your answer, denoted by $k,$ do not use $\epsilon_0.$\\

\begin{center}
\includegraphics[width=0.8\textwidth]{images/1734050275721_0.png}
\end{center}\\

Figure 1. A charged rod is bent into a circular segment with radius  $ a $ and subtends an angle of  $ \theta$. The rod has a charge density of  $ \lambda$.\\

 \\

\textbf{Part 1)}\\

\begin{center}
\includegraphics[width=0.8\textwidth]{images/1734050275721_1.png}
\end{center}\\

Figure 2. The rod can be broken up into infinitesimal segments. These segments subtend an angle of  $ d\phi$ with respect the point a distance  $ a $ away. The angle  $ \phi $ is the variable angle from the centre line.\\

Consider the increment shown in the figure. It subtends an angle $\mathrm{d} \phi \text{ rad}$ and is located $\phi$ radians clockwise from the centre. Write an expression describing the electric field at the point at the center of the circle substended by the arc due to this increment of change, note that $\mathrm{d} \phi$ appears after the answer box so should not be included in your expression.$$\\

$\mathrm{d} \mathbf{\vec{E}}_{\mathrm{d} \phi} =  ($  \underline{\hspace{3cm}}  $ \mathrm{d} \phi \mathbf{\hat{i}} + $  \underline{\hspace{3cm}}  $\mathrm{d} \phi \mathbf{\hat{j}} \text{ ) N/C}$ \\

\\

 \\

\textbf{Part 2)}\\

\begin{center}
\includegraphics[width=0.8\textwidth]{images/1734050275722_2.png}
\end{center}\\

 Figure 3. Using symmetry we can treat the two sides of the bent rod in pairs.\\

We will need to add all the increments around the arc. This is easier to do in pairs. What is the electric field at the point due to an increment $\mathrm{d} \phi$ considered in part 1 added to the increment $\mathrm{d} \phi, \phi \text{ rad}$ anticlockwise from the centre of the arc?\\

$\mathrm{d} \mathbf{\vec{E}}_{2 \mathrm{d} \phi} = ($  \underline{\hspace{3cm}}  $\mathrm{d} \phi \mathbf{\hat{i}} + $  \underline{\hspace{3cm}}  $\mathrm{d} \phi \mathbf{\hat{j}} \text{ ) N/C}$ \\

\\

 \\

\textbf{Part 3)}\\

By summing all the increments along the arc in pairs or otherwise, derive an expression for the electric field at the centre of the circle the arc lies upon.\\

$\mathbf{\vec{E}} = ($  \underline{\hspace{3cm}}  $ \mathbf{\hat{i}} + $  \underline{\hspace{3cm}}  $\mathbf{\hat{j}} \text{ ) N/C}$ \\

\\

\newpage

\textbf{Question:}\\

A hemisphere with a flat circular bottom is placed in an electric field as shown in the figure. In this case $\mathbf{\vec{E}} = {6.72 \times 10^{5}} \text{ N/C}$ and the radius of the hemisphere is $r = {5.52} \text{ cm.}$\\

\begin{center}
\includegraphics[width=0.8\textwidth]{images/1734050283933_0.png}
\end{center}\\

Figure 1. This is a 2D cross section of a flat quarter sphere of radius $r$ lying in an electric field of strength $\textbf{E}$.\\

\textbf{Part 1)}\\

What is the magnitude of the electric flux through the flat bottom of the hemisphere?\\

$ \Phi_E = $  \underline{\hspace{3cm}}  $\text{N m}^2 \text{/C}$ \\

\\

\textbf{Part 2)}\\

What is the magnitude of the electric flux through the top curved surface of the hemisphere?\\

$ \Phi_E = $  \underline{\hspace{3cm}}  $\text{N m}^2 \text{/C}$ \\

\\

\textbf{Part 3)}\\

What is the net flux through the entire enclosed hemisphere?\\

$ \Phi_E = $  \underline{\hspace{3cm}}  $\text{N m}^2 \text{/C}$ \\

\\

\newpage

\textbf{Question:}\\

Consider two protons separated by ${6.95} \times 10^{-15} \text{ m}$ ($10^{-15}$ m is a femto metre), this is a typical separation in a nucleus.\\

\textbf{Part 1)}\\

What is the magnitude of the Coulomb force between these protons?\\

$ F_C = $  \underline{\hspace{3cm}}  $\text{N}$ \\

\\

\textbf{Part 2)}\\

What is the gravitational force between these two protons?\\

$ F_G = $  \underline{\hspace{3cm}}  $\text{N}$ \\

\\

\textbf{Part 3)}\\

Assume that the protons are in equilibrium inside the nucleus. This requires another force, the strong nuclear force. What is the magnitude of this force if only the two protons are found in this nucleus?\\

$ F_S = $  \underline{\hspace{3cm}}  $\text{N}$ \\

\\

\newpage

\textbf{Question:}\\

Two identical conducting spheres are a distance $d = {1.09} \text{ cm}$ apart. Initially the sphere on the left has a charge of $+Q = + {2.47} \mu C$ and the sphere on the right is neutral. The sphere on the right is brought towards the one on the left so that it touches and then moved back to its original position. This is done carefully using an insulator.\\

\begin{center}
\includegraphics[width=0.8\textwidth]{images/1734050438467_0.png}
\end{center}\\

Figure 1. Two conducting (metal) spheres are placed a distance $d$ apart. The leftmost sphere has been charged to a value of $+Q$.\\

\textbf{Part 1)}\\

After this process what is the total charge on the system? (ie on the two spheres combined)?\\

$ Q = $  \underline{\hspace{3cm}}  $\mu \text{C}$ \\

\\

\textbf{Part 2)}\\

What is the charge on the sphere on the right after this process?\\

$ Q = $  \underline{\hspace{3cm}}  $\mu \text{C}$ \\

\\

\textbf{Part 3)}\\

How much work is done by the Coulomb force to move the sphere from $x = \frac{d}{2}$ back to its original position at $x = d?$ (Assume that $d$ is very much greater than the diameter of the spheres, so that charges are uniformly distributed.)\\

$ W = $  \underline{\hspace{3cm}}  $\text{J}$ \\

\\

Give a positive answer as the displacment and force are in the same direction.\\

\newpage

\textbf{Question:}\\

A non-conducting ring has a radius $r $ and a constant charge distribution of $\lambda $. In this question you are going to derive an expression for the electric field a distance $a $ above the centre of the ring. Use unit vector notation for your answers, the direction of the unit vectors is shown in the diagram. Use Coulomb's constant in your answer, denoted by $k,$ do not use $\epsilon_0.$\\

\begin{center}
\includegraphics[width=0.8\textwidth]{images/1734050454704_0.png}
\end{center}\\

Figure 1. A charged ring of radius  $ r $ and uniform charge density  $ \lambda$ sits on the  $ x, y $-axis. A point lies on the  $ z $-axis a distance  $ a $ above the ring. The ring can be broken into increments with length $ ds $. \\

\textbf{Part 1)}\\

Consider the increment shown in Figure 1 with length $\mathrm{d} s.$ IT is located at $r \mathbf{\hat{i}} \text{ m.}$ Write an expression to describe the electric field at the point due to the increment $\mathrm{d} s.$\\

$\mathbf{\vec{dE}}_{\mathrm{d} s} = ( $  \underline{\hspace{3cm}}  $\mathrm{d} s \mathbf{\hat{i}} + $  \underline{\hspace{3cm}}  $\mathrm{d} s \mathbf{\hat{j}} + $ \underline{\hspace{3cm}}  $\mathrm{d} s \mathbf{\hat{k}} \text{ ) N/C}$ \\

\\

\textbf{Part 2)}\\

\begin{center}
\includegraphics[width=0.8\textwidth]{images/1734050454706_1.png}
\end{center}\\

Figure 2. The ring can be broken up into infinitesimal segments   $ds$  and charge   $dq .$\\

We will need to add all the increments around this ring. This is easiest to do in pairs. What is the electric field at the point due to the pair of increments $\mathrm{d} s$ found at $r \mathbf{\hat{i}}$ and $-r \mathbf{\hat{i}}?$\\

$\mathbf{\vec{dE}}_{2 \mathrm{d} s} = ( $  \underline{\hspace{3cm}}  $\mathrm{d} s \mathbf{\hat{i}} + $  \underline{\hspace{3cm}}  $\mathrm{d} s \mathbf{\hat{j}} + $ \underline{\hspace{3cm}}  $\mathrm{d} s \mathbf{\hat{k}} \text{ ) N/C}$ \\

\\

\textbf{Part 3)}\\

By summing all the increments in pairs around the ring or otherwise, derive an expression for the electric field a distance $a \text{ m}$ above the centre of the ring.\\

$\mathbf{\vec{E}} = ($  \underline{\hspace{3cm}}  $ \mathbf{\hat{i}} + $  \underline{\hspace{3cm}}  $\mathbf{\hat{j}} + $ \underline{\hspace{3cm}}  $\mathbf{\hat{k}} \text{ N/C) }$ \\

\\

\newpage

\textbf{Question:}\\

A charge $q = + {8.44} \mu \text{C}$ is placed in the centre of a cube with side length $l = {4.84} \text{ cm.}$\\

\begin{center}
\includegraphics[width=0.8\textwidth]{images/1734050510399_0.png}
\end{center}\\

Figure 1. A cube of side length $a$ has a charge of $+q$ in its centre.\\

\textbf{Part 1)}\\

What is the net electric flux through the cube?\\

$ \Phi_E = $  \underline{\hspace{3cm}}  $\text{N m}^2 \text{/C}$ \\

\\

\textbf{Part 2)}\\

What is the electric flux through the top surface of the cube?\\

$ \Phi_E = $  \underline{\hspace{3cm}}  $\text{N m}^2 \text{/C}$ \\

\\

\textbf{Part 3)}\\

With the charge still in place, a constant electric field $\mathbf{\vec{E}} = {8.03 \times 10^{6}} \text{ N/C}$ is now set up going up the screen. What is the net electric flux through the entire cube now?\\

$ \Phi_E = $  \underline{\hspace{3cm}}  $\text{N m}^2 \text{/C}$ \\

\\

\newpage

\textbf{Question:}\\

Two positive charges with charge $q1=+8.96μCq_1 = +{8.96} \mu C$ and $q2=+7.58μCq_2 = +{7.58} \mu C$ are fixed in place a distance $d=22.24 cmd = {22.24} \text{ cm}$ apart. Charge $q1q_1$ is at the origin $x=0.x=0.$\\

\begin{center}
\includegraphics[width=0.8\textwidth]{images/1734050619214_0.png}
\end{center}\\

$Figure 1. Two positive charges, q1q_1 and q2q_2 are separated by a distance dd. $\\

\textbf{Part 1)}\\

What is the magnitude of the force that $q1q_1$ exerts on $q2?q_2?$\\

$F= F = $  \underline{\hspace{3cm}}  $N\text{N}$ \\

\\

\textbf{Part 2)}\\

A third charge, $q3=+1.16μCq_3 = +{1.16} \mu C$ is placed between $q1q_1$ and $q2.q_2.$ Where should this charge be placed in order for it to be in equilibrium?\\

$x= x = $  \underline{\hspace{3cm}}  $cm\text{cm}$ \\

\\

\textbf{Please note: Parts 3 and 4 have a combined mark of 0.33.}\\

\textbf{Part 3)}\\

When $+1.16μC,+ {1.16} \mu C,$ is placed at the equilibrium point, is the equilibrium stable or unstable? (Consider only motion in the x direction.)\\

$(No answer given)StableUnstable$ \\

\\

\textbf{Part 4)}\\

If the charge of the same magnitude but negative, ( $−1.16μC,- {1.16} \mu C,$ ) is placed at this same equilibrium position, would the equilibrium now be stable or unstable? (Consider only motion in the x direction.)\\

$(No answer given)StableUnstable$ \\

\\

\newpage

\textbf{Question:}\\

Four charges: $+q, -q,-2q \text{ and } +2q$ with $q = {1.60} \mu C$ are arranged around a square with sides of length $a = {8.58} \text{ cm}$ as shown in the figure. Use unit vector notation to answer the question.\\

\begin{center}
\includegraphics[width=0.8\textwidth]{images/1734050706060_0.png}
\end{center}\\

Figure 1. Four charges $+q$, $-q$, $+2q$ and $-2q$ are arranged in a square of side length $a$. The charges $+q$ and $=2q$ lie on opposite corners.\\

\textbf{Part 1)}\\

What force does the particle with charge $-2q$ exert on the particle with charge $+q?$\\

$\mathbf{\vec{F}} = $  \underline{\hspace{3cm}}  $ \mathbf{\hat{i}} + $  \underline{\hspace{3cm}}  $\mathbf{\hat{j}} \text{ N}$ \\

\\

\textbf{Part 2)}\\

What is the net force on the $+ q$ charge due to the other three forces?\\

$\mathbf{\vec{F}} = $  \underline{\hspace{3cm}}  $ \mathbf{\hat{i}} + $  \underline{\hspace{3cm}}  $\mathbf{\hat{j}} \text{ N}$ \\

\\

\textbf{Part 3)}\\

The $+q$ charge is now removed from the top left hand corner of the square. What is the electric field in the top left hand corner now?\\

$\mathbf{\vec{E}} = $  \underline{\hspace{3cm}}  $ \mathbf{\hat{i}} + $  \underline{\hspace{3cm}}  $\mathbf{\hat{j}} \text{ N/C}$ \\

\\

\newpage

\textbf{Question:}\\

A non conducting rod of length $l $ has a uniform charge per unit length of $\lambda $. In this question you are going to derive an expression for the electric field a distance $a$ above the centre of the rod. Use unit vector notation for your answers, the direction of the unit vectors is shown on the diagram. Use Coulomb's constant in your answer, denoted by $k,$ do not use $\epsilon_0.$\\

\begin{center}
\includegraphics[width=0.8\textwidth]{images/1734050756297_0.png}
\end{center}\\

Figure 1. A charged rod of length  $ l $ and uniform charge density  $ \lambda $ lies on the  $ x-$axis. A point lies a distance  $ a $ above the centre of the rod at $ x= 0 $.\\

 \\

\textbf{Part 1)}\\

\begin{center}
\includegraphics[width=0.8\textwidth]{images/1734050756297_1.png}
\end{center}\\

Figure 2. The rod can be broken up into infinitesimal segments  $ dx $. \\

Consider the increment shown in the figure. In a figure with length $\mathrm{d} x$ a distance $x$ to the right of the centre of the rod. Write an expression to describe the electric field at the point due to the increment $\mathrm{d} x.$\\

$\mathrm{d} \mathbf{\vec{E}}_{\mathrm{d} x} = ($  \underline{\hspace{3cm}}  $ \mathrm{d} x \mathbf{\hat{i}} + $  \underline{\hspace{3cm}}  $\mathrm{d} x \mathbf{\hat{j}} \text{ ) N/C}$ \\

\\

\textbf{Part 2)}\\

\begin{center}
\includegraphics[width=0.8\textwidth]{images/1734050756298_2.png}
\end{center}\\

Figure 3. The symmetry of the rod around the point allows us to simplify the problem.\\

We will need to add all the increments around the line. This is easier to do in pairs. What is the electric field at the point due to the pair of increments with length $\mathrm{d} x$ at $x$ and $-x$ along the rod?\\

$\mathrm{d} \mathbf{\vec{E}}_{2 \mathrm{d} x} = ( $  \underline{\hspace{3cm}}  $\mathrm{d} x \mathbf{\hat{i}} + $  \underline{\hspace{3cm}}  $\mathrm{d} x \mathbf{\hat{j}} \text{ ) N/C}$ \\

\\

\textbf{Part 3)}\\

By summing all the increments along the rod in pairs or otherwise, derive an expression for the electric field at the point a distance $ a$, above the centre of the rod.\\

$\mathbf{\vec{E}} = ($  \underline{\hspace{3cm}}  $ \mathbf{\hat{i}} + $  \underline{\hspace{3cm}}  $\mathbf{\hat{j}} \text{ ) N/C}$ \\

\\


\newpage

\textbf{Question:}\\

Consider a square with side length $l = {6.27} \text{ mm}$ placed in an electric field described by:\\

$ \mathbf{\vec{E}} = \left( {6.13 \times 10^{5}} + ({6.27 \times 10^{10}}) y^2 \right) \mathbf{\hat{k}} \text{ N/C.}$ where $y$ is in metres.\\

The vertices of the square are at $(0,0), (0,l), (l,0) \text{ and} (l,l)$ as shown on the diagram. Answer part 1 in unit vector notation, $\mathbf{\hat{k}}$ is out of the screen towards you.\\

 \\

\begin{center}
\includegraphics[width=0.8\textwidth]{images/1734050765080_0.png}
\end{center}\\

Figure 1. An Electric field in the $z$-direction is asymmetric in the $x,y$- plane. The density of electric field lines represents the relative strength of the electric field. In the $x$ direction this is constant but it increasing in the $y-$ direction.\\

\textbf{Part 1)}\\

What is the electric field in the centre of the square?\\

$\mathbf{\vec{E}} = $  \underline{\hspace{3cm}}  $ \mathbf{\hat{i}} + $  \underline{\hspace{3cm}}  $\mathbf{\hat{j}} + $ \underline{\hspace{3cm}}  $\mathbf{\hat{k}} \text{ N/C}$ \\

\\

\textbf{Part 2)}\\

Write a definite integral using $\mathrm{d} y$ but not $\mathrm{d} x$ that you could solve to calculate the electric flux through this square. Sub in variables and simplify your answer.\\

$\Phi_E = \int_{x_0}^{x_1} f ( y ) \mathrm{d} y. \text{ Nm}^2 \text{/C.}$\\

Where:\\

$x_0 = $  \underline{\hspace{3cm}}  \\

\\

$x_1 = $  \underline{\hspace{3cm}}  \\

\\

$ f ( y ) = $  \underline{\hspace{3cm}}  \\

\\

and $x_0 < x_1.$\\

\textbf{Part 3)}\\

Solve this integral to give the flux through the square.\\

$ \Phi_E = $  \underline{\hspace{3cm}}  \\

\\

\newpage

\textbf{Question:}\\

A non-conducting rod of length $l $ has a constant charge per unit length of $\lambda $ In this question you are going to derive an expression for the electric field a distance from the left hand end of the rod.\\

Use unit vector notation, the unit vectors are shown in the figure. Use Coulomb's constant in your answer, denoted by $k,$ do not use $\epsilon_0.$\\

 \\

\begin{center}
\includegraphics[width=0.8\textwidth]{images/1734050841918_0.png}
\end{center}\\

Figure 1. A uniformly charged rod of length  $ l $ and  linear charge density  $ \lambda$ lies on the  $ x- $  axis. The point lies a distance  $ a$ from the end of the rod.\\

\textbf{Part 1)}\\

Consider the increment shown in the figure with length $\mathrm{d} x$ a distance $x$ from the left hand end of the rod. Write an expression describing the electric field at the point a distance $a$ to the left of the end of the rod due to the increment $\mathrm{d}x.$\\

\begin{center}
\includegraphics[width=0.8\textwidth]{images/1734050841918_1.png}
\end{center}\\

Figure 2. The length of the rod can be broken down into infintesimal segments of  $ dx $.\\

Use Coulomb's constant in your answer, denoted by $k,$ do not use $\epsilon_0.$\\

$ \mathrm{d} \mathbf{\vec{E}}_{\text{dx}} = $  \underline{\hspace{3cm}}  $\mathrm{d} x \mathbf{\hat{i}} $ \\

\\

\textbf{Part 2)}\\

Now that you have an expression for one increment write down an integral expression that you could solve to give the total electric field of the rod. Do not solve the integral expression in this part.\\

$ \mathbf{\vec{E}} = \int_{x0}^{x1} f(x) \mathrm{d } x $\\

Where\\

$x_0 = $  \underline{\hspace{3cm}}  \\

\\

$x_1 = $  \underline{\hspace{3cm}}  \\

\\

$f(x) = $  \underline{\hspace{3cm}}  \\

\\

Note: $x_0 < x_1.$\\

\textbf{Part 3)}\\

Derive an expression for the electric field a distance, $ a $ from the left hand end of the rod.\\

$ \mathbf{\vec{E}} = $  \underline{\hspace{3cm}}  $\mathbf{\hat{i}} $ \\

\\

\newpage

\textbf{Question:}\\

An electric dipole consists of two identical but opposite charges separated by a bond of length $2a.$ Throughout this question the bond length does not change. An electric dipole with charges $+ {9.95} \mu \text{C}$ and $- {9.95} \mu \text{C}$ and separated by a distance ${1490} \text{ pm}$ is placed in a constant electric field $\mathbf{\vec{E}} = {5.26 \times 10^{5}} \mathbf{\hat{i}} \text{ N/C.}$ Initially the dipole is aligned along the $\mathbf{\hat{j}}$ axis as shown in the figure. Use unit vector notation to answer these questions. The $\mathbf{\hat{k}}$ unit vector is out of the screen towards you.\\

\begin{center}
\includegraphics[width=0.8\textwidth]{images/1734050998390_0.png}
\end{center}\\

Figure 1. Two equal but opposite charges are placed a distance $2a$ apart in the $y$ direction. In the $x$ direction a constant electric field $\textbf{E}$ is applied.\\

\textbf{Part 1)}\\

What is the force on the $+ {9.95} \mu C $ charge at the instant shown?\\

$\mathbf{\vec{F}} = $  \underline{\hspace{3cm}}  $ \mathbf{\hat{i}} + $  \underline{\hspace{3cm}}  $\mathbf{\hat{j}} + $ \underline{\hspace{3cm}}  $\mathbf{\hat{k}} \text{ N}$ \\

\\

\textbf{Part 2)}\\

What is the net force on the dipole in this electric field?\\

$\mathbf{\vec{F}} = $  \underline{\hspace{3cm}}  $ \mathbf{\hat{i}} + $  \underline{\hspace{3cm}}  $\mathbf{\hat{j}} + $ \underline{\hspace{3cm}}  $\mathbf{\hat{k}} \text{ N}$ \\

\\

\textbf{Part 3)}\\

What is the net torque on the dipole at the instant shown in the figure?\\

$\mathbf{\vec{\tau}} = $  \underline{\hspace{3cm}}  $ \mathbf{\hat{i}} + $  \underline{\hspace{3cm}}  $\mathbf{\hat{j}} + $ \underline{\hspace{3cm}}  $\mathbf{\hat{k}} \text{ Nm}$ \\

\\

\newpage

\textbf{Question:}\\

An electric dipole consists of two equal but opposite charges separated by a distance of $2a$ as shown in the diagram. In this question, give answers, in terms of $\epsilon_0,$ the permitivity of free space.\\

\begin{center}
\includegraphics[width=0.8\textwidth]{images/1734051037303_0.png}
\end{center}\\

Figure 1. Two equal but opposite charges $+q$ and $-q$ are placed a distance $2a$ apart to create an electric dipole.\\

\textbf{Part 1)}\\

Write an expression for the electric field at $x = r$ due to the $-q$ charge. The $-q$ charge is located at $x=a$ and in this case $r>a.$\\

$ \mathbf{\vec{E}} = $  \underline{\hspace{3cm}}  $\mathbf{\hat{i}} \text{ N/C} $ \\

\\

\textbf{Part 2)}\\

Write an expression for the total electric field at $x=r$ due to the $+q$ charge at $x=-a$ and the $-q$ charge at $x=a.$\\

$ \mathbf{\vec{E}} = $  \underline{\hspace{3cm}}  $\mathbf{\hat{i}} \text{ N/C} $ \\

\\

\textbf{Part 3)}\\

Now consider the limit where $r >> a.$ Write an expression for the electric field in this case.\\

$ \mathbf{\vec{E}} = $  \underline{\hspace{3cm}}  $\mathbf{\hat{i}} \text{ N/C} $ \\

\\

\newpage

\textbf{Question:}\\

A cube with side length $l = {4.33} \text{ cm}$ is placed in a constant electric field described by $\mathbf{\vec{E}} = {8.73 \times 10^{6}} \mathbf{\hat{j}} \text{ N/C.}$\\

\begin{center}
\includegraphics[width=0.8\textwidth]{images/1734051049456_0.png}
\end{center}\\

Figure 1. A cube of side length $a$ sits at the origin of the $x,y$-axis. A constant electric field is applied in the $y-$direction. Side 1 denotes the side where $z=0$ (pink with red edges) and Side 2 is the side where $y=l$ (pale blue).\\

\textbf{Part 1)}\\

What is the flux through the surface 1? Note that surface 1 has vertices at the origin, $l \mathbf{\hat{i}}, l\mathbf{\hat{i}} + l\mathbf{\hat{j}}$ and $l\mathbf{\hat{j}}.$\\

$ \Phi_E = $  \underline{\hspace{3cm}}  $\text{N m}^2 \text{/C}$ \\

\\

\textbf{Part 2)}\\

What is the electric flux through surface 2? Note that surface 2 has vertices at $l \mathbf{\hat{j}}, l\mathbf{\hat{i}} + l\mathbf{\hat{j}}, l \mathbf{\hat{i}} + l \mathbf{\hat{j}} - l \mathbf{\hat{k}}$ and $l\mathbf{\hat{j}} - l \mathbf{\hat{k}}.$\\

$ \Phi_E = $  \underline{\hspace{3cm}}  $\text{N m}^2 \text{/C}$ \\

\\

\textbf{Part 3)}\\

What is the net electric flux through the entire cube?\\

$ \Phi_E = $  \underline{\hspace{3cm}}  $\text{N m}^2 \text{/C}$ \\

\\

\newpage

\textbf{Question:}\\

Two charged balls both with mass $m$ and charge $q$ are suspended from a thread with length $l.$ They form a small angle $\theta,$ with the normal.\\

\begin{center}
\includegraphics[width=0.8\textwidth]{images/1734051063048_0.png}
\end{center}\\

Figure 1. Two charge of equal sign and magnitude $q$ and mass $m$ are connected to a massless string of length $l$. When hung from the same point on the ceiling the two charges repel each other and are separated by a distance $x$. Both masses are at an angle $\theta$  from the negative $y$-axis.\\

\textbf{Part 1)}\\

Use the small angle approximation to derive an expression for the angle $\theta$ (in radians) in terms of $\epsilon_0, l, q, m$ and $g.$ Do not include $x$ in your expression.\\

$ \theta = $  \underline{\hspace{3cm}}  $\text{radians}$ \\

\\

\textbf{Part 2)}\\

In the case where $l = {38.7} \text{ cm,}$ $q = {1.09} \mu C$ and $m = {46.8} \text{ g}$ what angle does the string make with the normal?\\

$ \theta = $  \underline{\hspace{3cm}}  $\text{radians}$ \\

\\

\textbf{Part 3)}\\

What is the separation, $x,$ in this case?\\

$ x = $  \underline{\hspace{3cm}}  $\text{cm}$ \\

\\




\end{document}